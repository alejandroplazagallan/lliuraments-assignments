\documentclass{article}
\usepackage[utf8]{inputenc}
\usepackage[T1]{fontenc}
\usepackage[catalan]{babel}
\usepackage{amsmath}
\usepackage{amssymb}
\usepackage{amsthm}
\usepackage[left=2cm,top=2.5cm,right=1.5cm,bottom=2.5cm]{geometry}

\newtheorem{teorema}{Teorema}
\newtheorem{corollari}{Corol\textperiodcentered lari}
\newtheorem{lema}{Lema}
\newtheorem{definicio}{Definici\'{o}}
\newtheorem{notacio}{Notaci\'{o}}
\newtheorem{proposicio}{Proposici\'{o}}
\newtheorem{observacio}{Observaci\'{o}}
\theoremstyle{definition}
\newtheorem{exemple}{Exemple}

\title{Ones viatgeres}
\author{Alejandro Plaza Gall\'{a}n}
\date{8 de juny de 2021}

\begin{document}

\maketitle
Hola, s\'{o}c l'Alejandro i far\'{e} una presentaci\'{o} sobre ones viatgeres.

Partim de l'\textbf{equaci\'{o} d'ones} unidimensional cl\`{a}ssica.

\[\frac{\partial^2u}{\partial t^2}-c^2\frac{\partial^2u}{\partial x^2}=0\]

Aquesta equaci\'{o} en derivades partials descriu la posici\'{o} dels punts d'una corda vibrant, on $u$ \'{e}s el despla\c{c}ament dins l'oscil\textperiodcentered laci\'{o}, $x$ \'{e}s la posici\'{o} dins la corda i $t$ \'{e}s el temps. Algunes solucions particulars d'aquesta equaci\'{o} s\'{o}n les funcions de la forma
\[\varphi(x-ct)\hspace{3mm}\text{o}\hspace{3mm}\varphi(x+ct),\]
on $\varphi$ \'{e}s una funci\'{o} $C^2$ qualsevol.

Les funcions d'aquesta forma es diuen \textbf{ones viatgeres}, perqu\`{e} a un temps $t$ fix tenim una funci\'{o} en $x$, i a temps $t+\Delta t$, tenim la mateixa funci\'{o} en $x$, per\`{o} avaluada en $x\pm c\Delta t$, \'{e}s a dir, la mateixa funci\'{o} per\`{o} despla\c{c}ada una dist\`{a}ncia $c\Delta t$. La velocitat a la que l'ona es propaga \'{e}s precisament $c$.

S\'{o}n molt importants perqu\`{e} amb aquestes funcions particulars (les ones viatgeres) es pot treure la soluci\'{o} general de l'equaci\'{o} d'ones.

El nostre objectiu ser\`{a} llavors trobar ones viatgeres com a solucions de diferents equacions diferencials de les quals no coneixem la soluci\'{o} general. Per exemple hi ha una generalitzaci\'{o} de l'equaci\'{o} de Fisher-Kolmogorov.

\[u_t=u_{xx}+f(u),\]
on
\begin{enumerate}
\item $f(0)=0$,
\item $f(1)=0$,
\item $f(u)>0$ per $0<u<1$,
\item $f''(u)<0$.
\end{enumerate}

L'equaci\'{o} de Fisher-Kolmogorov \'{e}s el cas particular que $f(u)=u(1-u)$. Busquem llavors solucions de la forma $u=U(x-ct)$. Volem una ona viatgera de tipus front, \'{e}s a dir, una ona com la de la imatge, que comenci l'$1$, decreixi cap a l'$0$ i es mogui cap a la dreta.
%Dibuix

Per aix\`{o} suposarem com a hip\`{o}tesi que $0\leq U\leq 1$ i que $c>0$.

\'{E}s clar que $u=0$ i $u=1$ s\'{o}n solucions trivials. Busquem llavors solucions no trivials, pel qual substitu\"{i}m $u$ per $U(x-ct)$ en l'equaci\'{o} de Fisher-Kolmogorov. Calculant
\[\frac{\partial u}{\partial t}=-cU'(x-c)\hspace{3mm}\text{i}\hspace{3mm}\frac{\partial^2u}{\partial x^2}=U''(x-c)\]
obtenim l'equaci\'{o}
\[U''+cU'+f(U)=0,\]
la qual \'{e}s una equaci\'{o} diferencial de segon ordre no lineal. La podem transformar en la EDO de primer ordre seg\"{u}ent:
\[\left\{\begin{array}{ll}U'=V,\\V'=-cV-f(U).\end{array}\right.\]

No se sap trobar la soluci\'{o} general d'aquesta EDO, tret del cas trivial $c=0$, en qu\`{e} queda un sistema hamiltoni\`{a}. Es pot demostrar anal\'{i}ticament que existeix una soluci\'{o} que d\'{o}na una ona viatgera, Per\`{o} nosaltres farem un estudi qualitatiu.

Els punts cr\'{i}tic d'aquest sistema els obtenim resolent $U'=0,V'=0$ i s\'{o}n el $P_1=(0,0)$ i $P_2=(1,0)$. Estudiem el tipus de punt cr\'{i}tic mitjan\c{c}ant el teorema de Hartman calculant la matriu jacobiana del camp $X$.

\[DX(U,V)=\left(\begin{matrix}0&1\\-f'(U)&-c\end{matrix}\right),\hspace{5mm}DX(0,0)=\left(\begin{matrix}0&1\\-f'(0)&-c\end{matrix}\right),\hspace{5mm}DX(1,0)=\left(\begin{matrix}0&1\\-f'(1)&-c\end{matrix}\right).\]

Estudiem els valors propis de $DX$ en el $(1,0)$. De les condicions de $f$ es dedueix que ha de ser $f'(1)<0$, implicant que $\det(DX(1,0))=f'(1)<0$. Per tant ha de tenir un valor propi positiu i una altre negatiu. Pel teorema de Hartman, el $(1,0)$ \'{e}s una sella.

Calculem els valors propis de $DX$ en $P_1$. El polinomi caracter\'{i}stic \'{e}s
\[\left|\begin{matrix}\lambda&-1\\f'(0)&\lambda+c\end{matrix}\right|=\lambda^2+c\lambda+f'(0).\]

Les arrels s\'{o}n
\[\lambda=\frac{-c\pm\sqrt{c^2-4f'(0)}}{2}.\]

Si el discriminant \'{e}s negatiu, \'{e}s a dir, si $|c|<2\sqrt{f'(0)}$, llavors els valors propis seran complexos i $P_1$ ser\`{a} un focus.

Si, en canvi, el discriminant \'{e}s no negatiu, o equivalentment, $|c|\geq2\sqrt{f'(0)}$, aleshores els valors propis seran reals. Si $c>0$, seran  valors propis negatius i tindrem un node atractor a $P_1$. Si $c<0$, seran negatius i $P_1$ ser\`{a} un node repulsor.

Observem que si $P_1=(0,0)$ \'{e}s un focus, aleshores la soluci\'{o} far\`{a} voltes al voltant de l'origen. Per tant $U$ prendr\`{a} valors negatius. Tanmateix, hav\'{i}em fet la hip\`{o}tesi que $0\leq U\leq1$. Per tant aquestes solucions no ens serveix. Aix\`{o} for\c{c}a a que $|c|\geq2\sqrt{f'(0)}$. A m\'{e}s, com que hav\'{i}em suposat que $c>0$, tenim que $c>2\sqrt{f'(0)}$. Per tant $P_1$ \'{e}s un node atractor.

Amb aix\`{o} tenim que hi haur\`{a} una ona viatgera de tipus front si i nom\'{e}s si hi ha una \`{o}rbita heterocl\'{i}nica. Aquesta haur\`{a} de n\'{e}ixer a la sella i morir al node atractor. Volem doncs calcular el $\omega$-l\'{i}mit de la varietat inestable $W^u(1,0)$ de la sella.

La separatriu inestable sortir\`{a} de la sella i volem confinar-la dins d'una regi\'{o} positivament inestable. Proposarem una regi\'{o} amb forma de lluna. Per brevetat fem el canvi de variable $x=1-U,y=V$ i escrivim $g(x)=-f(1-x)$. L'EDO ens queda:
\[\left\{\begin{array}{l}x'=-y,\\y'=-cy+g(x).\end{array}\right.\]

El retrat de fase d'aquest sistema \'{e}s com l'invers: la sella \'{e}s al $(0,0)$ i el node atractor al $(1,0)$. Prenem la fam\'{i}lia de corbes
\[G_{\lambda}(x,y)=y-\lambda g(x)=0,\]
de les quals trobarem dues que tancaran una regi\'{o} ---amb forma de lluna--- i seran tals que el camp sobre aquestes corbes apuntar\`{a} sempre cap a dins de la regi\'{o}. Comprovarem aix\`{o} \'{u}ltim fent el producte escalar de $G_{\lambda}$, el qual apunta cap a dalt, amb el camp $X$ sobre la corba $G_{\lambda}(x,y)=0$.

\[\langle\nabla G_{\lambda}(x,\lambda g(x)),X(x,\lambda g(x))\rangle=g(x)(1-c\lambda+g'(x)\lambda^2)=g(x)N_{\lambda}(x)\]

Com que $g(x)\leq0$, el signe de $\langle\nabla G_{\lambda},X\rangle$ ser\`{a} el contrari del de $N_{\lambda}(x)$. A m\'{e}s $N_{\lambda}'(x)=g''(x)\lambda^2>0$, pel qual $N_{\lambda}(x)$ \'{e}s creixent. Aix\'{i} doncs, si $N_{\lambda}(0)=0$, tindrem $N_{\lambda}(x)\geq0$ i si $N_{\lambda}(1)=0$, tindrem $N_{\lambda}(x)\leq0$. Siguin $\underline\lambda$ i $\overline\lambda$ tals que
\[N_{\underline\lambda}(0)=1-c\underline\lambda+g'(0)\underline\lambda^2=0,\]
\[N_{\overline\lambda}(1)=1-c\overline\lambda+g'(1)\overline\lambda^2=0.\]

Amb aix\`{o} llavors entre les corbes $G_{\underline\lambda}(x,y)=0$ i $G_{\overline\lambda}(x,y)=0$ es troba una regi\'{o} positivament invariant. Amb tot aix\`{o}, la separatriu inestable de la sella no podr\`{a} sortir de la regi\'{o}. Per tant, ha d'estar tota ella continguda a la regi\'{o} invariant. Ara, pel teorema de Poincar\'{e}-Bendixson, el seu $\omega$-l\'{i}mit no pot ser buit i per tant ha de contenir un punt cr\'{i}tic, que ser\`{a} el $(0,0)$.

Amb tot aquest argument hem vist que hi ha una \`{o}rbita heterocl\'{i}nica del sistema i per tant existeix una ona viatgera de l'equaci\'{o} en derivades parcials original.

Ara volem estudiar l'exist\`{e}ncia d'\`{o}rbites heterocl\'{i}niques pertorbant lleugerament l'equaci\'{o}. Per exemple podem considerar $f(u)=u(1-u)$ i pertorbar-la canviant-la per $u(1-u)(u-a)$ amb $0<a<1$.

\[\left\{\begin{array}{l}U'=V,\\V'=-cV-U(1-U)(U-a).\end{array}\right.\]

Ara tenim tres punts cr\'{i}tics: $(0,0)$, que ser\`{a} una sella, el $(a,0)$, que ser\`{a} un centre, node o focus, i el $(1,0)$, que ser\`{a} una altra sella. Per tant trobarem una connexi\'{o} entre les selles.
%Dibuix

Veurem que hi ha nom\'{e}s un $c>0$ pel qual existeix una connexi\'{o} heterocl\'{i}nica i, conseg\"{u}entment, una ona viatgera de tipus front.

\begin{definicio}
Una fam\'{i}lia de camps $X((x,y),c)=(P((x,y),c),Q((x,y),c))$ es diu \textbf{rotat\`{o}ria} quan
\[PQ_c-P_cQ\]
no canvia de signe i no \'{e}s id\`{e}nticament zero.
\end{definicio}

La motivaci\'{o} \'{e}s que $PQ_c-P_cQ$ t\'{e} el signe de la variaci\'{o} de l'angle del camp, com veiem a la seg\"{u}ent f\'{o}rmula.

\[\theta(c)=\arctan\frac{Q}{P};\hspace{7mm}\theta'(c)=\frac{PQ_c-P_cQ}{P^2+Q^2}.\]
(No descriure f\'{o}rmula)

Per tant en una fam\'{i}lia rotat\`{o}ria el camp sempre gira en el mateix sentit segons variant $c$. Aix\'{i}, si hi ha una connexi\'{o} heterocl\'{i}nica entre dues selles per un cert $c$, en pertorbar-lo una mica, s'ha de trencar la connexi\'{o} ja que les dues noves branques no poden creuar la antiga connexi\'{o}, com es veu al dibuix.
%Dibuix

Tornem a la nostra EDO
\[\left\{\begin{array}{l}U'=P((U,V),c)=V,\\V'=Q((U,V),c)=-cV-U(1-U)(U-a),\end{array}\right.\]
i calculem $PQ_c-P_cQ=-V^2<0$. Per tant \'{e}s una fam\'{i}lia rotat\`{o}ria i existir\`{a} com a molt un \'{u}nic $c$ pel qual hi hagi una connexi\'{o} heterocl\'{i}nica. Estudiem els casos l\'{i}mit per $c$.

Per $c=0$, el sistema \'{e}s potencial i per tant es pot realitzar f\`{a}cilment el retrat de fase sencer. En particular tindrem el seg\"{u}ent.
%Dibuix

Pel cas $c\to+\infty$, fent un canvi de variable a l'EDO, podem obtenir el seg\"{u}ent.
%Dibuix

Per tant, com que tenim una fam\'{i}lia rotat\`{o}ria, en algun $c$ es trobaran les dues separatrius de les dues selles i formaran una connexi\'{o}.
%Molts dibuixos

Per \'{u}ltim voldria comentar una mica per sobre l'obtenci\'{o} expl\'{i}cita d'ones viatgeres, i \'{e}s que tornant a l'equaci\'{o} de Fisher-Kolmogorov, si $c=\frac{5}{\sqrt{6}}$, per $k>0$, es va obtenir la soluci\'{o}
\[U(s)=\frac{1}{\left(1+ke^{\frac{s}{\sqrt{6}}}\right)^2},\]
que d\'{o}na lloc a una ona viatgera de velocitat $c=\frac{5}{\sqrt{6}}$.

S'han trobat tamb\'{e} d'altres ones viatgeres per altres equacions diferencials i totes elles tenen la forma
\[U(s)=\frac{q_1(e^{\lambda s})}{q_2(e^{\lambda s})},\]
on $\lambda\in\mathbb{R}$ i $q_1,q_2$ s\'{o}n polinomis reals.

\begin{definicio}
Diem que $U(s)$ \'{e}s una \textbf{soluci\'{o} algebraica} quan existeix un polinomi $p\in\mathbb{R}[x,y]$ no nul tal que $p(U(s),U'(s))=0$.
\end{definicio}

Fent servir conceptes algebraics com el resultant de dos polinomis per provar el seg\"{u}ent lema.

\begin{lema}
Si $U(s)$ \'{e}s una funci\'{o} racional amb variable exponencial, aleshores existeix un polinomi $p\in\mathbb{R}[x,y]$ tal que $p(U(s),U'(s))=0$.
\end{lema}

En particular, si tenim una soluci\'{o} d'aquesta forma, ser\`{a} una soluci\'{o} algebraica.
\end{document}
