\documentclass{beamer}
\usepackage[utf8]{inputenc}
\usepackage[T1]{fontenc}
\usepackage[catalan]{babel}
\usepackage{amsmath}
\usepackage{amssymb}
\usepackage{amsthm}
\usepackage{amscd}

\newtheorem{teorema}{Teorema}
\newtheorem{corollari}{Corol\textperiodcentered lari}
\newtheorem{lema}{Lema}
\newtheorem{definicio}{Definici\'{o}}
\newtheorem{notacio}{Notaci\'{o}}
\newtheorem{proposicio}{Proposici\'{o}}
\newtheorem{observacio}{Observaci\'{o}}
\theoremstyle{definition}
\newtheorem{exemple}{Exemple}

\mode<presentation>
{
\usetheme{Boadilla}
\usecolortheme{seahorse}
} 

\DeclareMathOperator{\Obj}{Obj}
\DeclareMathOperator{\Hom}{Hom}
\DeclareMathOperator{\id}{id}

\title{Topologia no commutativa}
\author{Alejandro Plaza Gall\'{a}n}
\date{8 de juny de 2021}

\begin{document}

\begin{frame}
\titlepage
\end{frame}

\begin{frame}{$C^*$-\`{a}lgebres}
Comencem recordant la definici\'{o} de $K$-\`{a}lgebra.
\pause

\begin{definicio}
Una $\boldsymbol{K}$\textbf{-\`{a}lgebra} \'{e}s un anell $A$ que \'{e}s a m\'{e}s espai vectorial sobre el cos $K$. Les estructures d'anell i espai vectorial s\'{o}n compatibles en el sentit seg\"{u}ent: per tots $\lambda\in K$, $a,b\in A$:
\begin{enumerate}
\item $\lambda(ab)=(\lambda a)b=a(\lambda b)=(ab)\lambda$,
\item $\lambda a=a\lambda$.
\end{enumerate}
\end{definicio}
\pause

En aquesta presentaci\'{o} tractarem de les $\mathbb{C}$-\`{a}lgebres.
\end{frame}

\begin{frame}{$C^*$-\`{a}lgebres}
\begin{exemple}
\begin{enumerate}
\item Les matrius $M_n(\mathbb{C})$ amb coeficients complexos \'{e}s una $\mathbb{C}$-\`{a}lgebra.
\pause
\item Donat un espai topol\`{o}gic compacte Hausdorff, podem definir $C(X)=C(X,\mathbb{C})=\{f:X\rightarrow\mathbb{C}\,|\,f\text{ es cont\'{i}nua}\}$, \pause que es pot dotar d'una estructura de $\mathbb{C}$-\`{a}lgebra natural: donades $f,g\in C(X)$, $\lambda\in\mathbb{C}$ i $x\in X$, definim
\[(f+g)(x)=f(x)+g(x),\]
\[(fg)(x)=f(x)g(x),\]
\[(\lambda f)(x)=\lambda f(x).\]

$C(X)$ \'{e}s de fet una $\mathbb{C}$-\`{a}lgebra commutativa.
\end{enumerate}
\end{exemple}
\end{frame}

\begin{frame}{$C^*$-\`{a}lgebres}
\begin{definicio}
Una \textbf{involuci\'{o}} sobre una $\mathbb{C}$-\`{a}lgebra \'{e}s una aplicaci\'{o} $*:A\rightarrow A$ tal que:
\begin{enumerate}
\item $(a+b)^*=a^*+b^*$,
\item $(ab)^*=b^*a^*$,
\item $(\lambda a)^*=\overline\lambda a^*$.
\end{enumerate}
\end{definicio}
\pause

\begin{exemple}
Sobre $C(X)$ podem agafar la involuci\'{o} consistent en prendre el conjugat: donada $f\in C(X)$, $f^*(x)=\overline{f(x)}$. En aquest cas a m\'{e}s $(fg)^*=f^*g^*$ perqu\`{e} aquesta \`{a}lgebra \'{e}s commutativa.
\end{exemple}
\end{frame}

\begin{frame}{$C^*$-\`{a}lgebres}

Sobre $C(X)$ definim la norma de manera que per $f\in C(X)$,
\[||f||=\sup_{x\in X}|f(x)|.\]

Aquesta norma existeix i \'{e}s finita gr\`{a}cies a que hem suposat que $X$ compacte.
\pause

\begin{definicio}
Una $\boldsymbol{C^*}$\textbf{-\`{a}lgebra} \'{e}s una $\mathbb{C}$-\`{a}lgebra $A$ amb una norma $||\cdot||:A\rightarrow[0,\infty)$ tal que $(A,||\cdot||)$ \'{e}s un espai m\`{e}tric complet, compleix $||ab||\leq||a||||b||$, t\'{e} una involuci\'{o} $*:A\rightarrow A$ i per tot $a\in A$ es compleix $||aa^*||=||a||^2$.
\end{definicio}
\pause

$C(X)$ \'{e}s una $C^*$-\`{a}lgebra.
\end{frame}

\begin{frame}{$C^*$-\`{a}lgebres}
\begin{definicio}
Un \textbf{morfisme de }$\boldsymbol{C^*}$\textbf{-\`{a}lgebres} $f:A\rightarrow B$ \'{e}s un morfisme d'anells tal que $f(a^*)=f(a)^*$.
\end{definicio}
\pause

\begin{teorema}
Si $f:A\rightarrow B$ \'{e}s un morfisme de $C^*$-\`{a}lgebres, aleshores per tot $a\in A$, $||f(a)||\leq||a||$. En particular, $f$ \'{e}s cont\'{i}nua.
\end{teorema}
\end{frame}

\begin{frame}{Espai topol\`{o}gic associat a una $C^*$-\`{a}lgebra}
\begin{definicio}
Sigui $A$ una $C^*$-\`{a}lgebra unit\`{a}ria commutativa. Definim l'\textbf{espai topol\`{o}gic associat} a $A$ com
\[M_A=\{f:A\rightarrow\mathbb{C}\,|\,f\text{ \'{e}s un morfisme d'anells }\mathbb{C}\text{-lineal},f\neq0\}.\]
\pause
Definim tamb\'{e} l'\textbf{espai dual topol\`{o}gic} de $A$ com
\[A^*=\{f:A\rightarrow\mathbb{C}\,|\,f\text{ \'{e}s }\mathbb{C}\text{-lineal i cont\'{i}nua}\}.\]
\end{definicio}
\pause

\begin{lema}
Sigui $A$ una $C^*$-\`{a}lgebra unit\`{a}ria commutativa. Sigui $\varphi\in M_A$. Llavors:
\begin{enumerate}
\item $\displaystyle{||\varphi||=\sup_{||a||\leq1}|\varphi(a)|=1}$.
\item $\varphi(a^*)=\overline{\varphi(a)}$.
\end{enumerate}
\end{lema}
\end{frame}

\begin{frame}{Espai topol\`{o}gic associat a una $C^*$-\`{a}lgebra}
\begin{definicio}
Dotem $M_A$ de la topologia indu\"{i}da per ser subespai de $A^*$. Aquesta es l'anomenada \textbf{topologia d\`{e}bil*}. \'{E}s la topologia en la que una successi\'{o} $\varphi_n\in M_A$ convergeix a una $\varphi\in M_A$ si i nom\'{e}s si convergeix puntualment.
\end{definicio}
\pause

\begin{lema}
$M_A\subseteq A^*$ \'{e}s un espai compacte.
\end{lema}
\pause

\begin{definicio}
Sigui $A$ una $C^*$-\`{a}lgebra unit\`{a}ria commutativa. Definim la \textbf{transformada de Gelfand} com l'aplicaci\'{o}
\[\begin{split}A&\longrightarrow C(M_A),\\a&\longmapsto\hat{a}\end{split}\hspace{7mm}\text{on}\hspace{7mm}\begin{split}\hat{a}:M_A&\longrightarrow\mathbb{C}.\\\varphi&\longmapsto\varphi(a)\end{split}\]
\end{definicio}
\end{frame}

\begin{frame}{Espai topol\`{o}gic associat a una $C^*$-\`{a}lgebra}
\begin{teorema}
(Gelfand) Sigui $A$ una $C^*$-\`{a}lgebra unit\`{a}ria commutativa. Llavors la transformada de Gelfand $A\rightarrow C(M_A)$ \'{e}s un *-morfisme isom\`{e}tric.
\end{teorema}
\pause

El teorema de Gelfand ens ha perm\`{e}s trobar per qualsevol $C^*$-\`{a}lgebra unit\`{a}ria commutativa $A$ un espai topol\`{o}gic $X=M_A$ compacte Hausdorff tal que $C(M_A)\cong A$. La aplicaci\'{o} que identifica $C(M_A)$ amb $A$ \'{e}s la transformada de Gelfand.
\end{frame}

\begin{frame}{Llenguatge de categories}
\begin{definicio}
Una \textbf{categoria} $\mathcal{C}$ \'{e}s una estructura algebraica que consta de
\begin{enumerate}
\item una clase $\Obj(\mathcal{C})$ d'\textbf{objectes};
\item per $C,D\in\Obj(\mathcal{C})$, un conjunt $\Hom_{\mathcal{C}}(C,D)$ de \textbf{morfismes} de $C$ a $D$;
\item per cada $C,D,E\in\Obj(\mathcal{C})$, una aplicaci\'{o}
\begin{align*}
\Hom_{\mathcal{C}}(C,D)\times\Hom_{\mathcal{C}}(D,E)&\longrightarrow\Hom_{\mathcal{C}}(C,E).\\
(f,g)&\longmapsto fg
\end{align*}
\end{enumerate}
\pause

A m\'{e}s exigim les seg\"{u}ents propietats:
\begin{enumerate}
\item Per tot $C\in\Obj(\mathcal{C})$ existeix un $\id_C\in\Hom_{\mathcal{C}}(C,C)$.
\item Per tots $C,D\in\Obj(\mathcal{C})$ i $f\in\Hom_{\mathcal{C}}(C,D)$, $\id_Df=f\id_E=f$.
\item Per tots $C,D,E,F\in\Obj(\mathcal{C})$ i $f\in\Hom_{\mathcal{C}}(C,D)$, $g\in\Hom_{\mathcal{C}}(D,E),h\in\Hom_{\mathcal{C}}(E,F)$, es compleix $(fg)h=f(gh)$.
\end{enumerate}
\end{definicio}
\end{frame}

\begin{frame}{Llenguatge de categories}
\begin{exemple}
\begin{enumerate}
\item Hi ha la categoria dels espais topol\`{o}gics compactes Hausdorff $CH$. Els seus morfismes s\'{o}n les aplicacions cont\'{i}nues entre ells.
\pause
\item Tamb\'{e} hi ha la categoria de les $C^*$-\`{a}lgebres commutatives unit\`{a}ries $C_{\text{ab},1}$, amb els seus morfismes.
\end{enumerate}
\end{exemple}
\end{frame}

\begin{frame}{Llenguatge de categories}
\begin{definicio}
Un \textbf{functor} $F:\mathcal{C}\rightarrow\mathcal{D}$ entre dues categories $\mathcal{C},\mathcal{D}$ \'{e}s una correspond\`{e}ncia tal que per cada $C\in\Obj(\mathcal{C})$, tenim $F(C)\in\Obj\mathcal{D}$ i per tots $C,C'\in\Obj(\mathcal{C})$ i tot $f\in\Hom_{\mathcal{C}}(C,C')$, tenim $F(f)\in\Hom_{\mathcal{D}}(F(C),F(C'))$. Exigim les seg\"{u}ents propietats:
\pause
\begin{enumerate}
\item Per tot $C\in\Obj(\mathcal{C})$, $F(\id_C)=\id_{F(C)}$.
\item Per tots $C,C',C''\in\Obj(\mathcal{C})$ i tots $f\in\Hom_{\mathcal{C}}(C,C'),g\in\Hom_{\mathcal{C}}(C',C'')$, hi ha dues opcions:
\begin{enumerate}
\item O b\'{e} $F(fg)=F(f)F(g)$. En aquest cas $F$ es diu \textbf{covariant}.
\item O b\'{e} $F(fg)=F(g)F(f)$. En aquest cas $F$ es diu \textbf{contravariant}.
\end{enumerate}
\end{enumerate}
\end{definicio}
\end{frame}

\begin{frame}{Llenguatge de categories}
\begin{exemple}
Tenim el functor
\begin{align*}
F:CH&\longrightarrow C_{\text{ab},1}^*\\
X&\longmapsto C(X)
\end{align*}
que sobre els morfismes actua de manera que donats $X,Y\in CH$ i $f:X\rightarrow Y$ cont\'{i}nua,
\begin{align*}
F(f):C(Y)&\longrightarrow C(X).\\
\alpha&\longmapsto\alpha\circ f
\end{align*}
Observem que aquest functor \'{e}s contravariant, ja que inverteix l'ordre.
\end{exemple}
\end{frame}

\begin{frame}{Llenguatge de categories}
\begin{exemple}
Tamb\'{e} tenim el functor
\begin{align*}
G:C_{\text{ab},1}^*&\longrightarrow CH\\
A&\longmapsto M_A
\end{align*}
que sobre els morfismes actua de manera que donades $A,B\in C_{\text{ab},1}^*$ i $f:A\rightarrow B$ morfisme de $C^*$-\`{a}lgebres,
\begin{align*}
G(f):M_B&\longrightarrow M_A.\\
\varphi&\longmapsto\varphi\circ f
\end{align*}
Aquest functor tamb\'{e} \'{e}s contravariant.
\end{exemple}
\end{frame}

\begin{frame}{Llenguatge de categories}
Hem vist que $F(G(A))=C(M_A)\cong A\cong A$. Per tant, d'alguna manera $F\circ G$ \'{e}s com el functor identitat. Tamb\'{e} tenim que $G(F(X))=M_{C(X)}\cong X$, de manera que $G\circ F$ \'{e}s com el functor identitat. Vegem llavors en quina manera $F$ i $G$ s\'{o}n m\'{u}tuament inversos.
\pause

\begin{definicio}
Donades $\mathcal{C},\mathcal{D}$ categories i $F,G:\mathcal{C}\rightarrow\mathcal{D}$ functors, una \textbf{tranformaci\'{o} natural} de $F$ a $G$, que es denota com $\eta:F\Rightarrow G$, \'{e}s una col\textperiodcentered lecci\'{o} de morfismes $\eta_C:F(C)\rightarrow G(C)$ indexat sobre $C\in\Obj(\mathcal{C})$ que compleix que per tots $C,C'\in\Obj(\mathcal{C})$ i tot $f\in\Hom_{\mathcal{C}}(C,C')$, el seg\"{u}ent diagrama \'{e}s commutatiu:
\[\begin{CD}F(C)@>\eta_C>>G(C)\\@VF(f)VV@VVG(f)V\\F(C')@>>\eta_{C'}>G(C).\end{CD}\]
\end{definicio}
\end{frame}

\begin{frame}{Llenguatge de categories}
\begin{definicio}
Una transformaci\'{o} natural $\eta:F\Rightarrow G$ entre functors s'anomena \textbf{equival\`{e}ncia natural} si per tot $C\in\Obj(\mathcal{C})$, $\eta_C:F(C)\rightarrow G(C)$ \'{e}s un isomorfisme. Ho denotem per $F\simeq G$.
\end{definicio}
\pause

\begin{definicio}
Diem que dues categories s\'{o}n \textbf{naturalment equivalents} si existeixen functors $F:\mathcal{C}\rightarrow\mathcal{D}$ i $G:\mathcal{D}\rightarrow\mathcal{C}$ covariants tals que $G\circ F\simeq\id_{\mathcal{C}}$ i $F\circ G\simeq\id_{\mathcal{D}}$. Ho denotem com $\mathcal{C}\simeq\mathcal{D}$.

Si $F$ i $G$ s\'{o}n contravariants, diem que les $\mathcal{C}$ i $\mathcal{D}$ s\'{o}n \textbf{duals}.
\end{definicio}
\begin{teorema}
$CH\simeq C_{ab,1}^*$.
\end{teorema}
\end{frame}
\end{document}
